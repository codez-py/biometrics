\chapter{INTRODUCTION}
\graphicspath{{Chapter1/}}

\section{Gender Classification}
\subsection{Existing approaches}
Gender is an important fundamental attribute of humans. Nowadays, most gender recognition methods are based on facial features. This does not perform well when the subject is far away from cameras. And when the resolution of the cameras is low.

\subsection{Using Gait}
So in order to overcome the above discussed issues, we are using GAIT analyses in order to find the gender of a human.

\section{Gait Analysis}
Gait analysis, the study of human walking patterns, has become a useful bio metric technique for identification and recognition tasks. One important gait representation called Gait Energy Images (GEIs) has gained popularity. GEIs are averaged silhouette images over a full gait cycle that capture a person's gait pattern in a compact image. 

However, GEIs are typically captured from a lateral side view, which may not always be possible in real-world scenarios. This has led to research into generating GEIs from arbitrary viewing angles to make the approach more practical.

Gender classification from gait is an important preliminary task that can improve the performance of subsequent identification systems. This project proposes a method to accurately classify an individual's gender based solely on their gait patterns captured from any viewing angle using GEIs.

The proposed approach combines the strengths of GEIs in representing gait with advanced machine learning techniques to achieve accurate gender classification from certain observation angles.
